\documentclass[a4paper]{article}

%use the english line for english reports
%usepackage[english]{babel}
\usepackage[portuguese]{babel}
\usepackage[utf8]{inputenc}
\usepackage{indentfirst}
\usepackage{graphicx}
\usepackage{verbatim}


\begin{document}

\setlength{\textwidth}{16cm}
\setlength{\textheight}{22cm}

\title{\Huge\textbf{Título do Trabalho}\linebreak\linebreak\linebreak
\Large\textbf{Relatório Intercalar}\linebreak\linebreak
\includegraphics[height=6cm, width=7cm]{feup.pdf}\linebreak \linebreak
\Large{Mestrado Integrado em Engenharia Informática e Computação} \linebreak \linebreak
\Large{Programação em Lógica}\linebreak
}

\author{\textbf{Grupo 13:}\\ Hugo Ari Rodrigues Drumond - 201102900 \\ João Alexandre Gonçalinho Loureiro - 200806067 \\\linebreak\linebreak \\
 \\ Faculdade de Engenharia da Universidade do Porto \\ Rua Roberto Frias, s\/n, 4200-465 Porto, Portugal \linebreak\linebreak\linebreak
\linebreak\linebreak\vspace{1cm}}
%\date{Junho de 2007}
\maketitle
\thispagestyle{empty}

%************************************************************************************************
%************************************************************************************************

\newpage

\section*{Resumo}
O jogo de tabuleiro que iremos desenvolver chama-se Epaminondas. As peças presentes no tabuleiro são todas
iguais e movem-se como o Rei no xadrez, exceto quando se agrupam numa linha, coluna ou diagonal. Trata-se
do 1º. trabalho de grupo, da disciplina de Programação em Lógica, realizado pelo Grupo 13.
%Descrever muito sumariamente (1-2 parágrafos) o trabalho que está a ser reportado

%************************************************************************************************
%************************************************************************************************

%*************************************************************************************************
%************************************************************************************************

\section{Introdução}
O fim deste trabalho é criar um jogo de tabuleiro com base na matéria exposta nas aulas teóricas e téorico-práticas.
Este trabalho é interessante porque obriga-nos a pensar de um ponto de vista puramente lógico, isto é, cria-se uma base de dados
lógica através de cláusulas, sem controlo de fluxo (explícito), e depois procuramos por possíveis soluções. Este paradigma de programação
chama-se programação declarativa.

%Descrever os objectivos e motivação do trabalho.
%Todas as figuras devem ser referidas no texto. %\ref{fig:codigoFigura}


%Exemplo de código para inserção de figuras
%\begin{figure}[h!]
%\begin{center}
%escolher entre uma das seguintes três linhas:
%\includegraphics[height=20cm,width=15cm]{path relativo da imagem}
%\includegraphics[scale=0.5]{path relativo da imagem}
%\includegraphics{path relativo da imagem}
%\caption{legenda da figura}
%\label{fig:codigoFigura}
%\end{center}
%\end{figure}


%\textbf{Info útil}:

%Devem ser incluídas referências bibliográficas correctas e completas (consultar os docentes em caso de dúvida). Páginas da wikipedia não são consideradas referências válidas \cite{CodigoSite, CodigoLivro}.

%\textit{Para escrever em itálico}

%\textbf{Para escrever em negrito}

%Para escrever em letra normal

%``Para escrever texto entre aspas''

%Para fazer parágrafo, deixar uma linha em branco.

%Como fazer bullet points:

%\begin{itemize}
%\item Item1
%\item Item2
%\end{itemize}

%Como enumerar itens:

%\begin{enumerate}
%\item Item 1
%\end{enumerate}

%\begin{quote}``Isto é uma citação''\end{quote}

\section{O Jogo Epaminondas}
Este jogo, inicialmente chamado \textit{Crossings} e jogado num tabuleiro 8x8, foi descrito pela primeira vez no Livro \textit{A Gamut of Games} em 1969.
Depois da publicação deste livro, Bob Abbott, reconfigurou o tabuleiro de 8x8 para 14x12, de modo a tornar o jogo nas diagonais mais importante, e renomeou
o nome para Epaminondas como homenagem ao general TheBan que inventou a phalanx, uma formação de guerra usada em 371 a.c para derrotar o exército espartano.\cite{creator,rules}
\\\linebreak
Neste jogo, são posicionadas 28 peças iguais em cada lado maior do tabuleiro, com cores diferentes. São as brancas a iniciar o jogo, e depois joga-se alternadamente. O objetivo do jogo e ter mais peças que o adversário na linha mais recuada da formação adversária inicial, ou seja, a linha final. Os movimentos de cada peça são iguais ao do Rei do xadrez, pode andar uma casa em qualquer direção. Podem ser criados inúmeros grupos de peças, phalanxes, estas só se podem deslocar um número igual ou menor ao número de peças do grupo, na direcção da linha formada pelo grupo(para frente ou para trás). Uma peça pode fazer parte de várias phalanxes. Uma phalanx pode ser dividida, mas só pode andar um número de casas igual ou menor ao numero de elementos da phalanx que se irá mover. Não é permitido passar um jogada, em cada jogada é obrigatório mover uma phalanx ou uma peça. Todas as peças têm de estar contidas no tabuleiro. Não podem haver peças sobrepostas ou na mesma posição, exceto quando a cabeça de uma phalanx captura outra peça ou phalanx. Para capturar é necessário atacar com um grupo com mais elementos que o adversário na linha de ataque. A peça/phalanx capturada é removida do tabuleiro para sempre. \cite{rules}


%Descrever sucintamente o jogo, a sua história e, principalmente, as suas regras. Devem ser criadas/utilizadas imagens apropriadas para explicar o funcionamento do jogo.

\section{Representação do Estado do Jogo}
Descrever a forma de representação do estado do tabuleiro (tipicamente uma lista de listas), com exemplificação em Prolog de posições iniciais do jogo, posições intermédias e finais, acompanhadas de imagens ilustrativas.

\section{Visualização do Tabuleiro}
Descrever a forma de visualização do tabuleiro em modo de texto e os predicados Prolog construídos para o efeito. O código (predicado) desenvolvido deve receber como parâmetro a representação do tabuleiro (estado do jogo) e permitir visualizá-lo no ecrã, em modo de texto. Deve ser incluída pelo menos uma imagem correspondente ao output produzido pelo predicado de visualização

\section{Movimentos}
Elencar os movimentos (tipos de jogadas) possíveis e definir os cabeçalhos dos predicados que serão utilizados (ainda não precisam de estar implementados).

\section{Conclusões e Perspectivas de Desenvolvimento}
Que conclui da análise do jogo e da pesquisa bibliográfica realizada? Como vai ser desenvolvido o trabalho? Que parte (\%) do trabalho estima que falta fazer?

\clearpage
\addcontentsline{toc}{section}{Bibliografia}
\renewcommand\refname{Bibliografia}
\bibliographystyle{plain}
\bibliography{myrefs}

\newpage
\appendix
\section{Nome do Anexo}
Código Prolog implementado (representação do estado, cabeçalhos dos predicados para as jogadas e predicado que permite a visualização em modo de texto do tabuleiro).

\end{document}
